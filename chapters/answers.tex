\chapter*{Answers to exercises}\label{answers}
\addcontentsline{toc}{chapter}{Answers to exercises}
\ohead{Answers to exercises}

\section*{Chapter 1}

\noindent\textbf{\exerciseref{sower-and-seed}}\label{sower-answers}

\noindent Here we only mention a few salient features, including those suggested in the hints.
\begin{itemize}
    \item Phonetics and phonology: <seed> vs. <seedis> vs. <sæd> suggests differences in the first vowel,\is{vowels} but also the difference of one as opposed to two syllables in one case
    \item Morphology: endings such as -\emph{ith} (third person singular) and -\emph{en} (third person plural)\is{plurals} in the Wycliffe version, which are no longer found today. There are also archaic word forms like \emph{yede} (Wycliffe)\ia{Wycliffe, John} and \emph{eode} (West Saxon Gospels),\il{Old English, West Saxon} an old past tense form of the verb \emph{GO}.
    \item Syntax: the New International version has \emph{was scattering}, using the progressive\is{progressive} (see §\ref{LModE-progressive} for details), which the earlier versions do not have.
    \item Lexicon: there are some different word choices in the earlier versions, such as \emph{DEVOUR} (a verb of \ili{French} origin) rather than \emph{EAT} in the King James\ia{Stuart, James (King James I)} version. The King James version and the West Saxon Gospels\il{Old English, West Saxon} both use a form of \emph{FOWL} rather than \emph{BIRD} -- in Present Day English, \emph{FOWL} is very restricted in usage to a particular type of bird, and only used in technical contexts.\is{birds}
    \item Pragmatics/discourse:\is{historical pragmatics} the earlier versions have discourse markers\is{discourse markers} like \emph{Behold} and \emph{Lo!} that are not used in Present Day English (or at least not unless specifically trying to evoke an archaic tone). In speech, the modern words \emph{Hey} and \emph{So} have a roughly similar function at the beginning of a statement, though they don't tend to be used in writing in this way.
\end{itemize}

\noindent\textbf{\exerciseref{jabberwocky}}
\begin{enumerate}
    \item \begin{itemize}
        \item \emph{brillig}, \emph{mome}: noun or adjective
        \item \emph{slithy}, \emph{mimsy}, \emph{fruminous}, \emph{uffish}, \emph{tulgey}: adjectives
        \item \emph{toves}, \emph{borogoves}, \emph{wabe}, \emph{raths}: nouns
        \item \emph{gyre}, \emph{gimble}, \emph{outgrabe}, \emph{whiffing}, \emph{burbled}: verbs
        \item \emph{Jabberwock}, \emph{Jubjub}, \emph{Bandersnatch}: proper nouns
    \end{itemize}
    \item \begin{itemize}
        \item -\emph{y}: e.g. \emph{lucky}, \emph{happy}
        \item -\emph{ous}: e.g. \emph{obvious}, \emph{luminous}
        \item -\emph{ish}: e.g. \emph{ticklish}, \emph{Spanish}
        \item -(\emph{e})\emph{s}: most plural\is{plurals} nouns (e.g. \emph{houses}, \emph{doors})
        \item -\emph{ing}: present participles (e.g. \emph{sleeping}, \emph{running})
        \item -\emph{ed}: most past tense verbs (e.g. \emph{waited}, \emph{opened})
    \end{itemize}
\end{enumerate}

\noindent\textbf{\exerciseref{exercise-syntax}}
\begin{enumerate}
    \item \emph{Everyone}
    \item \emph{bumblebees}\is{bumblebees}
    \item \emph{should}
    \item \emph{appreciate bumblebees} (the whole VP)
    \item IP
    \item \emph{should} (the head I) and \emph{appreciate bumblebees} (the whole VP)
    \item \emph{appreciate} (the head V)
\end{enumerate}

\noindent\textbf{\exerciseref{exercise-variation}}\

\noindent Below we provide suggested answers to the first of the four questions:

\noindent{Q1:}
\begin{enumerate}
    \item B, D, H
    \item depending on who you ask also E: even standards are subject to change
\end{enumerate}

\noindent{Q2:}
\begin{enumerate}
    \item A: double \textit{from}
    \item C: \textit{dunno}, \textit{wi}; these are seen as non-standard when spelt, but in spoken language most of us actually pronounce these in a reduced way without even noticing!
    \item C and E: some people would also say that \textit{kind of}, \textit{sort of}, \textit{you know}, and \textit{sort of thing} are not standard; they are definitely more informal and colloquial
    \item E: the preposition in \textit{laughed upon} would be considered archaic and occasionally even non-standard by some
    \item F: \textit{divvent}, negative concord\is{negation} \textit{nt} + \textit{nowt}
    \item G: \textit{us} used with a singular referent; negative concord\is{negation} \textit{never} + \textit{no}
    \item H: lexical \textit{hadn't} (rather than \textit{didn't have}) is archaic and would be considered non-standard by some
    \item I: \textit{fur-sty} (this is known as TH-fronting)
\end{enumerate}

\section*{Chapter 2}

\noindent\textbf{\exerciseref{exercise-practice-features}}
\begin{enumerate}
    \item Quotative \emph{BE like}\is{quotatives}
    \item Singular verb agreement with collective noun\is{collective nouns} (\emph{team is})
    \item New words associated with social media:\is{media} \emph{photobomb} and \emph{selfie} (the original tweet also contains two emoji)
    \item \emph{GET}-passive (\emph{got arrested})
    \item Absence of third-person singular -\emph{s}\is{third person -\emph{s}} (on \emph{like} and \emph{want})
\end{enumerate}

\noindent\textbf{\exerciseref{exercise-ngrams}}
\begin{itemize}
    \item [A.]\begin{enumerate}
        \item The word \textit{fuck} is found with fluctuating frequency before 1820, then disappears until the 1950s, when it starts to rapidly increase in frequency.\is{frequency}
        \item Texts from before 1820 are often printed with the long s, <ſ>, in some contexts rather than <s> -- including word-initially. Optical character recognition (OCR) software struggles to distinguish <ſ> from <f>. Therefore, these early apparent instances of \emph{fuck} may in fact be instances of \emph{suck}. See for instance the text in §\ref{EModE-Robinson-text}, where a \emph{ſucking child} is mentioned.
        \item Another possibility to consider is a potential semantic change in \textit{fuck}: if a word has different meanings, or even connotations, in different periods, it wouldn't be surprising to see different frequencies\is{frequency} at which it's used.
    \end{enumerate}
    \item [B.]\begin{enumerate}
        \item It depends on the period.
        \item In the corpus\is{corpora} ``English (2012)'', the word \emph{biscuit} is more frequent until about 1977.
        \item In the corpus ``American English (2012)'',\il{English, American} \emph{biscuit} is overtaken by \emph{cookie} earlier, around 1973. In the corpus ``British English (2012)'',\il{English, British} \emph{biscuit} remains more frequent, though \emph{cookie} increases in frequency\is{frequency} throughout the 20th century.
        \item The word \emph{cookie} is an innovative Americanism.\is{Americanisms}
    \end{enumerate}
    \item [C.] Use your own initiative here.
    \item [D.] It's difficult to tell whether \textit{BE like} and \textit{BE all} increase over time, though it seems as if they do increase from about 1965 onwards. In British English,\il{English, British} the increase seems to start somewhat later, in the 1970s. One problem with this search is that \textit{BE like} and \textit{BE all} are hard to search for: you need to search for all word forms in the \glossterm{gl-paradigm}{paradigm}\is{paradigms} of \emph{BE}. Both forms remain vanishingly rare compared to \emph{SAY} according to this data.  
    \item[E.] The problem with searching for the \emph{GET}-passive\is{passive} is that it is a syntactic construction, and the N-gram Viewer is set up to search for words, not for constructions. You can try to search for common combinations (e.g. \emph{get paid}, \emph{got fired}), but there are many possible combinations of a form of \emph{GET} and a participle, and it would take forever to search for all of them.
    \item[F.] Google Books is a corpus\is{corpora} of books. Books represent literary language, not (on the whole) spoken language. (This is indeed most likely why we're getting surprising results for \textit{BE like} in E.) We also don't know what sort of authors are represented in the corpus,\is{corpora} and whether this is stably balanced over time. (For instance, there are probably many more women authors from the 20th century than from the 17th.)
    \item[G.] Again, this one is up to you and your imagination!
\end{itemize}

\section*{Chapter 3}
\noindent\textbf{\exerciseref{exercise-boldly}}

\noindent At least two reasons suggest themselves:

\begin{itemize}
    \item \textit{No man} is less inclusive than \textit{no one} in terms of gender.\is{gender studies}
    \item \textit{No man} is also less inclusive in terms of species. Star Trek is full of a range of beings!
\end{itemize}

\noindent\textbf{\exerciseref{exercise-backtranslate}}
\begin{quote}
    \emph{I love} the look of the new house, but it is still \emph{building}, so I \emph{should not} go in there yet, even though I \emph{want to}. I would probably \emph{be} hit by falling bricks and \emph{cry out} ``Owwww\footnote{Or \textit{Good Lord!}}, this \emph{was not} a good idea!'' I \emph{must} be patient.
\end{quote}


\noindent\textbf{\exerciseref{exercise-Amer}}\is{Americanisms|(}

\noindent Words restricted to North America: \textit{butte}, \textit{cougar}, \textit{woodchuck}. Comparing the information in the OED and your own familiarity with how these are used may give you different results. Words restricted to specific regions\is{regional variation} in North America: \textit{butte}, \textit{woodchuck}. First attestation and etymology:
\begin{itemize}
    \item \textit{to antagonise}, 1634 (OED, 2020, \glossterm{gl-sv}{s.v.} antagonize, v.), from \ili{Greek} (criticised by Brits as an Americanism in the past)
    \item \textit{to belittle}, 1785 (OED, 2020, \glossterm{gl-sv}{s.v.} belittle, v.), coined by Thomas Jefferson (criticised by Brits as an Americanism in the past)
    \item \textit{butte}, 1805 (OED, 2020, \glossterm{gl-sv}{s.v.} butte, n.), from \ili{French}
    \item \textit{coca-cola}, 1887 (OED, 2020, \glossterm{gl-sv}{s.v.} Coca-Cola, n.), information provided for the individual words, not the compound as such
    \item \textit{cookie}, 1754 (OED, 2020, \glossterm{gl-sv}{s.v.} cookie, n.), from \ili{Dutch}
    \item \textit{cougar}, 1774 (OED, 2020, \glossterm{gl-sv}{s.v.} cougar, n.), from \ili{French}, possibly ultimately from \ili{Guarani}
    \item \textit{creek}, 1300 (OED, 2020, \glossterm{gl-sv}{s.v.} creek, n.1), from \ili{French} but also \ili{Dutch}
    \item \textit{funky}, 1680 (OED, 2020, \glossterm{gl-sv}{s.v.} funky, adj.1), origin of \textit{funk} is uncertain, but \textit{funky} first occurs in the context of Buddy Bolden's Blues in the US
    \item \textit{lengthy}, 1759 (OED, 2020, \glossterm{gl-sv}{s.v.} lengthy, adj.), Americanism
    \item \textit{woodchuck}, 1670 (OED, 2020, \glossterm{gl-sv}{s.v.} woodchuck, n.), from Algonquian, probably \ili{Ojibwe}\is{Americanisms|)}
\end{itemize}


\noindent\textbf{\exerciseref{exercise-semchange}}\is{amelioration|(}\is{pejoration|(}

\noindent Here we only mention which cases could be analysed as cases of amelioration and/or pejoration. These are simplistic answers, and you'll have to engage with the OED to reach more detailed and more critical interpretations.


\begin{itemize}
    \item \textit{baboon}: may be a case of pejoration, but we don't see a clear time-line progression whereby the more negative meanings would be attested in later stages (2020, \glossterm{gl-sv}{s.v.} baboon, n.)
    \item \textit{churl}: pejoration (2020, \glossterm{gl-sv}{s.v.} churl, n.)
    \item \textit{gay}: pejoration (2020, \glossterm{gl-sv}{s.v.} gay, adj., adv., and n.); but also amelioration, in the context of the reclamation of the term by the gay community
    \item \textit{girl}: some instances of pejoration are attested after the word changed from its original meaning, `a child of either sex', to refer to `a female human' (2020, \glossterm{gl-sv}{s.v.} girl, n.)
    \item \textit{hussy}: pejoration (2020, \glossterm{gl-sv}{s.v.} hussy | huzzy, n.)
    \item \textit{mouse}: amelioration and pejoration depending on the meaning  (2020, \glossterm{gl-sv}{s.v.} mouse, n.)
    \item \textit{weed}: pejoration (2020, \glossterm{gl-sv}{s.v.} weed, n.1)
    \item \textit{artful}: pejoration (2020, \glossterm{gl-sv}{s.v.} artful, adj. (and n.))
    \item \textit{coy}: pejoration (2020, \glossterm{gl-sv}{s.v.} coy, adj.)
    \item \textit{crafty}: pejoration (2020, \glossterm{gl-sv}{s.v.} crafty, adj.)
    \item \textit{cunning}: pejoration (2020, \glossterm{gl-sv}{s.v.} cunning, adj.)
    \item \textit{fond}: amelioration (2020, \glossterm{gl-sv}{s.v.} fond, adj. and n.)
    \item \textit{funky}: amelioration (2020, \glossterm{gl-sv}{s.v.} funky, adj.1)
    \item \textit{jolly}: pejoration (2020, \glossterm{gl-sv}{s.v.} jolly, adj. and adv.)
    \item \textit{lewd}: pejoration (2020, \glossterm{gl-sv}{s.v.} lewd, adj.)
    \item \textit{nice}: amelioration (2020, \glossterm{gl-sv}{s.v.} nice, adj. and adv.)
    \item \textit{shrewd}: amelioration (2020, \glossterm{gl-sv}{s.v.} shrewd, adj.)
    \item \textit{silly}: pejoration (2020, \glossterm{gl-sv}{s.v.} silly, adj., n., and adv.)
    \item \textit{subtle}: complicated (2020, \glossterm{gl-sv}{s.v.} subtle, adj. and n.)
    \item \textit{to await}: amelioration (2020, \glossterm{gl-sv}{s.v.} await, v.)
    \item \textit{to shit}: pejoration (2020, \glossterm{gl-sv}{s.v.} shit, v.)
    \item \textit{to smite}: complicated (2020, \glossterm{gl-sv}{s.v.} smite, v.)
    \item \textit{to tease}: complicated, but limiting ourselves to OED only pejoration, otherwise also amelioration (2020, \glossterm{gl-sv}{s.v.} tease, v.1)\is{amelioration|)}\is{pejoration|)}
\end{itemize}


\section*{Chapter 4}

\noindent\textbf{\exerciseref{ex-GVS}}\is{Great Vowel Shift}

\noindent Simplistic answers are provided here, and you may well disagree with them for a range of reasons:\\

\noindent A.
\begin{itemize}
    \item diphthongal: Greater Manchester, Kent, London, North Yorkshire, Co. Tipperary
    \item monophthongal: Newcastle, Orkney, Perthshire
\end{itemize}

\noindent B.\\
\noindent Monophthongal realizations are more conservative.\\


\noindent\textbf{\exerciseref{ex-thou}}

\noindent1.
\begin{itemize}
    \item A. Attorney General switches from the respectful \textit{you} form (V form) to the less respectful \textit{thou} forms (T forms).
    \item B. Hamlet addresses the Ghost with \textit{thou} forms until he realizes the Ghost is his father -- he momentarily switches to the more respectful \textit{you}.
\end{itemize}



\noindent2. They are in line with the use of the pronouns.\is{pronouns} For instance, Attorney General addresses Raleigh as \textit{thou traitor} in the passage in which he switches to the less respectful forms.\\

\noindent3., 4. Use your own imagination here.\\


\noindent\textbf{\exerciseref{exercise-shakespeare-syntax}}

\noindent See §\ref{EModE-do} if you need to check the notions covered in these answers.\is{\emph{DO}-support}

\begin{enumerate}
    \item \emph{What saies he}: This shows movement of the lexical verb from V to I to C in an interrogative,\is{V-to-I movement} since the finite verb \emph{saies} precedes the subject\is{subjects} \emph{he}. In Present Day English, we would insert \emph{DO} here: \emph{What does he say?} (or, with the progressive,\is{progressive} \emph{What is he saying?}).
    \item \emph{you know not}: \emph{you do not know}, with \emph{DO}-support in a negative declarative.\is{negation}
    \item \emph{why call you}: \emph{why do you call}, as in the first example.
    \item \emph{saw you him}: \emph{did you see him}, with \emph{DO}-support in an interrogative.
    \item \emph{Fear me not}: Present Day English \emph{Do not fear me} (or \emph{Don't fear me}). This is an imperative, which we didn't discuss in the chapter, but the principle is the same as for the other types of clause we've discussed: lexical verbs can't move from V to I to C any more, and instead \emph{DO} must be inserted.
\end{enumerate}

\noindent\textbf{\exerciseref{exercise-synonymity}}

\noindent In the below, the originally Germanic word is given on the left, and the rough \ili{Latin} or Romance equivalent on the right.

\begin{multicols}{2}
\begin{itemize}
    \item \textit{bless}: \textit{consecrate}
    \item \textit{break}: \textit{disintegrate}
    \item \textit{chew}: \textit{masticate}
    \item \textit{drink}: \textit{imbibe}
    \item \textit{flood}: \textit{inundate}
    \item \textit{free}: \textit{emancipate}
    \item \textit{go}: \textit{depart}
    \item \textit{job}: \textit{position}
    \item \textit{lie}: \textit{prevaricate}
    \item \textit{think}: \textit{cogitate}
\end{itemize}
\end{multicols}

\noindent All the Germanic words given here are monosyllabic, and all the \ili{Latin}/Romance words are polysyllabic. In some cases, the \ili{Latin}/Romance words are semantically narrower: for instance, to \emph{disintegrate} something is to \emph{break} it in a very specific way, into tiny pieces. In general, the words on the right are found more often in formal registers.\\

\noindent\textbf{\exerciseref{exercise-irregular-plurals}}\is{irregularities}\is{plurals}

\begin{itemize}
    \item \textit{analysis}: \emph{analyses} (\ili{Greek}, via \ili{Latin})
    \item \textit{cherub}: \emph{cherubim} (\ili{Hebrew}, via \ili{Greek} and \ili{Latin})
    \item \textit{index}: \emph{indices} (\ili{Latin})
    \item \textit{matrix}: \emph{matrices} (\ili{Latin}/\ili{French})
    \item \textit{medium}: \emph{media} (\ili{Latin})
    \item \textit{nucleus}: \emph{nuclei} (\ili{Latin})
    \item \textit{species}: \emph{species} (\ili{Latin})
    \item \textit{stigma}: \emph{stigmata} (\ili{Greek}, via \ili{Latin})
    \item \textit{stratum}: \emph{strata} (\ili{Latin})
\end{itemize}

\noindent\textbf{\exerciseref{EModE-inkhorn-exercise}}\is{inkhorn controversy}

\noindent The following words are of Germanic origin: \textit{clean}, \textit{borrow}, \textit{tongue}, \textit{never}, \textit{keep}, \textit{house}; as are the following affixes:\is{affixes} \textit{-ed}, \textit{-ing}, \textit{un-}. So Cheke\ia{Cheke, John} himself uses plenty of borrowings.\is{borrowings}


\section*{Chapter 5}

\noindent\textbf{\exerciseref{ex-North}}

\noindent The take-home message is this:
\begin{itemize}
    \item We don’t have uniform results in any of the levels (not within lexical variables, not within phonological variables, not within syntactic variables). It follows that variation found in one level doesn’t have to be reflected in the other levels of (a/the) language.
    \item The important thing to realize here is that the data can be very messy and whilst some variables may be very or fairly clear indicators of a divide of some sort (such as the North-South divide), others may indicate divides in a more subtle way, and yet others may not really indicate anything in terms of regional differences.\is{regional variation} Variation is messy more often than not.
\end{itemize}


\noindent\textbf{\exerciseref{ex-Reeve}}

\noindent On the whole, Chaucer\ia{Chaucer, Geoffrey} is indeed consistent in assigning features typical of southern ME to the Miller and the Narrator, and those typical of northern ME to Aleyn and John. But see lines 4066, 4072, and 4073 for the southern character using the third person plural\is{plurals} pronoun\is{pronouns} \textit{they}.\\

We can only speculate what happened here:
\begin{itemize}
    \item Perhaps \textit{they} had creapt into the south by this stage, at least to some extent?
    \item Maybe this comes from a manuscript\is{manuscripts} copied by a speaker natively speaking a more northern variety of Middle English than Chaucer did?
\end{itemize}

\noindent In any case, Chaucer is suspiciously consistent here. As is obvious from the previous exercise, variation tends to be more on the messier side.\\


\noindent\textbf{\exerciseref{exercise-ME-pronouns}}
\begin{multicols}{2}
\begin{enumerate}
    \item \emph{s(c)he}\is{pronouns}
    \item \emph{ye}
    \item \emph{hie}
    \item \emph{thou}
    \item \emph{thei}
    \item \emph{ye}
    \item \emph{I} or \emph{ich}
    \item \emph{thei}
    \item \emph{he(o)}
\end{enumerate}
\end{multicols}


\noindent\textbf{\exerciseref{exercise-chaucer-V2}}
\begin{enumerate}
    \item \begin{itemize}
        \item (\ref{ex:chaucer-V2-a}): strict V2\is{verb-second}
        \item (\ref{ex:chaucer-V2-b}): either strict V2 or IS-V2
        \item (\ref{ex:chaucer-V2-c}): IS-V2
        \item (\ref{ex:chaucer-V2-d}): either strict V2 or IS-V2
        \item (\ref{ex:chaucer-V2-e}): IS-V2
        \item (\ref{ex:chaucer-V2-f}): strict V2
    \end{itemize}
    \item All the examples from \emph{Astrolabe} are strict V2 or compatible with a strict V2 syntax. All the examples from the \emph{Parson's Tale} are IS-V2 or compatible with IS-V2 syntax.
    \item \citet{Eitler2006} argues that variation in Chaucer's\ia{Chaucer, Geoffrey} syntax can be explained by the audience he is writing for. Chaucer\ia{Chaucer, Geoffrey} himself was mainly a user of a strict V2 variety, and uses this when writing for himself and those close to him (as in \emph{Astrolabe}, for his son), but when writing for a wider audience -- as in the \emph{Canterbury Tales} -- he uses much more IS-V2.\is{verb-second}
\end{enumerate}

\noindent\textbf{\exerciseref{exercise-wardrobe}}
\begin{itemize}
    \item \textit{guardian}:\is{lexical doublets} someone who guards (and often has legal responsibility, as in \textit{parent or guardian})\\
    \textit{warden}: a watchman or official
    \item \textit{garderobe}: a storeroom or toilet\\
    \textit{wardrobe}: a cupboard for storing clothes
    \item \textit{guard} (noun): someone who protects/watches over\\
    \textit{ward} (noun): someone or something protected
    \item \textit{guarantee}: an assurance\\
    \textit{warranty}: security, a legal\is{legal language} assurance\\
    \textit{warrant}: authorization, permission\is{lexical doublets}
\end{itemize}


\noindent\textbf{\exerciseref{ex-Owl}}
\begin{enumerate}
    \item \textit{his} vs. \textit{her}; \textit{his} is not feminine, but \textit{her} is. The former relies on grammatical gender, the latter on biological sex.\is{gender (grammatical)}
    \item \textit{his} vs. \textit{her} -- see above; \textit{eyre} vs. \textit{eggs}: here we see different plural\is{plurals} strategies (\textit{-re} vs. \textit{-s}).
    \item Present Day English uses a loan\is{borrowings} from \ili{Latin}.
    \item semantic change: Middle English \textit{brides} refers to young birds,\is{birds} whereas Present Day English \textit{birds} refers to winged creatures generally (hence \textit{chicks}).
    \item semantic change: Middle English \textit{mete} refers to food generally, rather than edible flesh, as is the case in Present Day English (hence \textit{meat}).
    \item /h/-dropping\is{/h/-dropping} (\textit{hit} vs. \textit{it}); Middle English shows a prefix\is{affixes} (\textit{i-}), unlike Present Day English.
    \item two different verbs are used: the Middle English one is strong,\is{strong verbs} the Present Day English one is weak.
    \item Middle English thorn\is{thorn} vs. Present Day English <th>; \textit{wilde} shows an ending, \textit{wild} doesn't; \textit{bowe} shows an ending, \textit{branch} doesn't; \textit{bowe} and \textit{branch} are two different words.
    \item different words: Present Day English shows a loan\is{borrowings} from \ili{French}.
\end{enumerate}


\noindent\textbf{\exerciseref{ex-Gawain}}
\begin{enumerate}
    \item Middle English shows thorn,\is{thorn} which is gone today; we see spelling differences in \textit{ceased} -- today's version shows the effects of the Great Vowel Shift,\is{vowels} not visible in the spelling\is{orthography}
    \item the Middle English spelling suggests that the second syllable contains an /u/ and carries stress, unlike the situation in Present Day English
    \item Middle English shows thorn,\is{thorn} which is gone today; Present Day English \textit{earth} does not have an inflectional\is{inflection} suffix,\is{affixes} but the Middle English version does
    \item inflectional\is{inflection} suffix present in Middle English
    \item inflectional\is{inflection} suffix present in Middle English
    \item inflectional\is{inflection} suffix present in the Middle English noun; the verbal suffix\is{affixes} must have contained an unstressed vowel\is{vowels} in Middle English
\end{enumerate}

\noindent\textbf{\exerciseref{ex-statements-ME}}

\noindent 
\begin{itemize}
    \item cause and effect problem: sound change happened first here and this became reflected in the spelling\is{orthography}
    \item it's true the first sound is a dental fricative, but it makes no sense to call it ``definite'' (or ``indefinite''!)
    \item bad logic: both are lexical words; they are not related
\end{itemize}

\newpage
\section*{Chapter 6}

\noindent\textbf{\exerciseref{exercise-OE-spelling}}
\begin{multicols}{2}
\begin{enumerate}
  \item ash\is{orthography}
  \item bed
  \item kin
  \item day
  \item fish
  \item hill
  \item man
  \item might
  \item over
  \item ship
  \item thorn
  \item thorn
  \item thin
\end{enumerate}
\end{multicols}

\noindent\textbf{\exerciseref{exercise-OE-pronounce}}

[ʍæt weː gaːrdena in jæːɑrdaɣum]

\noindent Don't worry if you put [g] instead of [ɣ] in the final word -- we didn't cover this in the book. See \citet[91]{Hogg1992} for details of this voiced velar fricative.\\

\noindent\textbf{\exerciseref{exercise-OE-fricatives}}

\begin{itemize}
    \item Voiceless: \textbf{s}æ, wi\textbf{f}, \textbf{f}lota, cea\textbf{s}ter, ly\textbf{f}t, yrh\textbf{ð}o, of\textbf{f}rian, oð\textbf{ð}e, wæ\textbf{s}, \textbf{þ}egn, be-\textbf{s}ecgan
    \item Voiced: bro\textbf{ð}or, hræ\textbf{f}n, weor\textbf{þ}e, o\textbf{f}er, ri\textbf{s}an, æ\textbf{þ}ele, bo\textbf{s}om, e\textbf{f}ne, no\textbf{s}u, bli\textbf{þ}e, heo\textbf{f}on, hæ\textbf{s}len
\end{itemize}

\noindent\textbf{\exerciseref{exercise-OE-nominalmorph}}
\begin{enumerate}
    \item Accusative, singular, masculine, strong
    \item Nominative or accusative, plural,\is{plurals} neuter, strong
    \item Nominative, singular, feminine, strong
    \item Genitive, singular, masculine or neuter (in fact it's masculine -- but you can only know this by looking at a dictionary\is{dictionaries} or at other forms), weak
    \item Genitive, plural,\is{plurals} any gender\is{gender (grammatical)} (in fact it's neuter -- but you can only know this by looking at a dictionary or at other forms), weak
    \item Accusative, singular, masculine, weak
    \item Accusative, singular, feminine, strong
    \item Dative, plural,\is{plurals} any gender (in fact it's masculine -- but you can only know this by looking at a dictionary or at other forms), weak or strong (in fact it's strong -- but you can only know this by looking at a dictionary\is{dictionaries} or at other forms)
\end{enumerate}

\noindent\textbf{\exerciseref{exercise-OE-murder}}
\begin{enumerate}
    \item The \emph{fruma} (chief).
    \item The \emph{hlæfdige} (lady).
    \item The \emph{fruma} was making unwanted advances, including using violence.
\end{enumerate}

\noindent\textbf{\exerciseref{exercise-OE-plurals}}\\
\noindent A. No -- the morpheme -\emph{as} was just one of many ways of forming the plural\is{plurals} in Old English, and was restricted to the strong masculine nouns. See §\ref{OE-case}.\\

\noindent B.

\begin{enumerate}
  \item \textit{sheep} $\sim$ \textit{sheep}: strong neuter long
  \item \textit{man} $\sim$ \textit{men}: athematic
  \item \textit{deer} $\sim$ \textit{deer}: strong neuter long
  \item \textit{fish} $\sim$ \textit{fish}: strong neuter long (actually, in Old English \textit{fish} was a strong masculine noun, but ran over to another paradigm\is{paradigms} camp later on)
  \item \textit{knife} $\sim$ \textit{knives}: strong masculine
  \item \textit{ox} $\sim$ \textit{oxen}: weak
  \item \textit{wolf} $\sim$ \textit{wolves}: strong masculine
\end{enumerate}

\noindent C.

\begin{enumerate}
  \item \textit{tooth} $\sim$ \textit{teeth}: \emph{i}-umlaut (see §\ref{OE-umlaut})
  \item \textit{louse} $\sim$ \textit{lice}: \emph{i}-umlaut
  \item \textit{child} $\sim$ \textit{children}: this is a trickier one (sorry!); \textit{-en} indicates a weak noun, but there is actually also a plural\is{plurals} \textit{-r} affix, representing a marginal class not shown in the magic sheet; today's \textit{children} have historically two plural\is{plurals} suffixes!\is{affixes}
\end{enumerate}


\noindent\textbf{\exerciseref{ex-OE-PDE}}

\noindent Only the main differences are commented on here:
\begin{enumerate}
    \item Old English lacks a reflexive pronoun\is{pronouns} (\textit{him} vs. \textit{himself})
    \item Old English shows no article;\is{articles} \textit{solitary} is a loan\is{borrowings} from \ili{French}; Old English shows an ending (\textit{-a})
    \item Object Verb in Old English, Verb Object in Present Day English; Old English has a different verbal ending than Present Day English; Old English again has a suffix\is{affixes} on the noun (\textit{-e})
    \item Old English shows no articles;\is{articles} Old English shows the genitive suffix \textit{-es} where Present Day English uses the preposition \textit{of}
    \item Old English shows verbal endings (\textit{-e}, \textit{-an}) where Present Day English has none; <sc> corresponds to <sh>; /hr/ is gone in Present Day English
    \item Old English has a dative plural\is{plurals} suffix (\textit{-um}) where Present Day English has a suffix\is{affixes} marking only number (\textit{-s})
    \item Old English uses a ligature\is{ligature} which is no longer present in today's English (also, notice the effects of the Great Vowel Shift)\is{vowels}\is{Great Vowel Shift}
    \item V2\is{verb-second} in Old English; Old English shows no article;\is{articles} the eth\is{eth} grapheme is not used in Present Day English; the noun has an ending in Old English (\textit{-a})
    \item what is expressed through the preposition \textit{of} in Present Day English is expressed through an ending in Old English (\textit{-a})
\end{enumerate}


\noindent\textbf{\exerciseref{exercise-OE-VPs}}
\begin{enumerate}
    \item ``But he must reconcile the quarrelsome'' (head-final)
    \item ``that he can do no good'' (head-final)
    \item ``he wanted to kill King David'' or ``who wanted to kill that king David'' (head-initial)
    \item ``if she would tolerate that disgrace'' (head-final)
    \item ``that he wanted to reach his hut'' (head-initial)
\end{enumerate}


\noindent\textbf{\exerciseref{ex-OE-statements}}

\noindent
\begin{enumerate}
    \item ``gendered'' means something different than ``having grammatical gender''\is{gender (grammatical)}
    \item adverbs don't have cases\is{case}
    \item English has descended from Germanic, as have various other languages (such as \ili{Swedish})
    \item a verb in the infinitive can't have tense; a verbal ending cannot function as a plural\is{plurals} marker of nouns
    \item ``casing'' means something different than ``morphologically marked cases''\is{case}
    \item it was inherited from Germanic
    \item Old English did not have a free word order; V2\is{verb-second} does not mean that the finite verb is going to show up at the end of a sentence\is{word order}
    \item verbal mood morphology is not determined by nouns
    \item thorn\is{thorn} is a letter and has nothing to do with morphemes (or phonemes, for that matter)
\end{enumerate}

\section*{Chapter 7}

\noindent\textbf{\exerciseref{exercise-reconstruction}}\is{reconstruction}
\begin{multicols}{3}
\begin{enumerate}
    \item *slāpan
    \item *skip
    \item *mūsi
    \item *skāp
    \item *hand
    \item *skīnan
\end{enumerate}
\end{multicols}

\noindent\textbf{\exerciseref{exercise-futhark}}
\begin{multicols}{2}
\begin{enumerate}
    \item Old Frisian\il{Frisian, Old} is fun\is{runes}
    \item I see dead people
    \item The truth is out there
    \item One small step for man
\end{enumerate}
\end{multicols}

\newpage
\noindent\textbf{\exerciseref{exercise-grimm}}\is{First Sound Shift}
\begin{multicols}{3}
\begin{enumerate}
    \item \emph{pedal}: foot
    \item \emph{genu}: knee
    \item \emph{dens}: tooth
    \item \emph{cor}: heartily
    \item \emph{piscis}: fish
    \item \emph{granum}: corn
    \item \emph{glaces}: cool
    \item \emph{dicere}: teach
    \item \emph{cornu}: heart
    \item \emph{tres}: three
    \item \emph{genus}: kin
    \item \emph{edo}: eat
\end{enumerate}
\end{multicols}

\noindent\textbf{\exerciseref{exercise-umlaut}}\is{umlaut}
\begin{enumerate}
    \item *gōs-i (plural\is{plurals} noun) > \emph{gēs}: /oː/ becomes /øː/ by \emph{i}-umlaut. Subsequently the final /i/ in the unstressed syllable is lost.\is{vowel reduction} The /øː/ later unrounds to /eː/. 
    \item *fōd-jan > \emph{fēdan}: /oː/ becomes /øː/ by \emph{i}-umlaut. Subsequently the /j/ in the unstressed syllable is lost. The /øː/ later unrounds to /eː/.
    \item *stel-idi > \emph{stilþ}: /e/ becomes /i/ by \emph{i}-umlaut. Subsequently the vowels\is{vowels} in the unstressed syllables are lost. (Don't worry about the details of the changes in the unstressed syllables here.)
\end{enumerate}

\noindent\textbf{\exerciseref{exercise-regular-irregular}}\is{strong verbs}\is{weak verbs}
\begin{enumerate}
    \item Weak; regular
    \item Strong; irregular. It's a class V strong verb. The stem vowel\is{vowels} alternation in the \textsc{2sg.pres} form is caused by \emph{i}-umlaut.\is{umlaut}
    \item Strong; regular
    \item Weak; irregular. The giveaway is the coronal /t/ consonant\is{consonants} in the past tense ending, and the fact that the stem vowel\is{vowels} /oː/ in this form does not correspond to any of the strong verb classes. The sound changes that lead to this irregularity are complex and we won't discuss them here. Interestingly, forms like \emph{bringst} (\textsc{2sg.pres}) and \emph{brang} (\textsc{3sg.past}) are also attested, suggesting that some speakers had reanalysed the verb as a class III strong verb.
\end{enumerate}

\section*{Chapter 8}

\noindent\textbf{\exerciseref{sower-and-seed-2}}\\
See the \hyperref[sower-answers]{answers} to the Chapter 1 version of this exercise.
